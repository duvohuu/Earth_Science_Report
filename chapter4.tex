\chapter{ATMOSPHERE}

    \section{Describe the characteristics of storms in Vietnam}

        \hspace*{0.6cm}Vietnam is located in the Northwest Pacific region — one of the most active tropical cyclone basins in the world. With a coastline stretching over 3,200 km, the country is directly affected by many typhoons forming in the East Sea (South China Sea).

        \subsection{Number and Season of Storms}

        \hspace*{0.6cm}According to the National Center for Hydro-Meteorological Forecasting (NCHMF, 2023), an average of 10–12 tropical cyclones form in the East Sea each year, of which about 5–7 directly affect Vietnam’s mainland.  
        The typhoon season typically lasts from June to November, peaking between September and October.

        \begin{figure}[H]
            \centering
            \includegraphics[width=0.65\textwidth]{pictures/chapter4/c4_p1_Latitude.png}
            \caption{Frequency of storms making landfall in Vietnam by decade (1977–2017)\protect\footnotemark}
        \end{figure}
        \footnotetext{Takagi \& Yoshida (2019). ``Statistics on Typhoon Landfalls in Vietnam''. \textit{ResearchGate}.}

        \subsection{Intensity and Scale of Storms}

        \hspace*{0.6cm}Storms affecting Vietnam usually reach wind levels of 8–11 (equivalent to 62–117 km/h), and in many cases reach level 13–14 (130–160 km/h), which are considered very strong by regional standards.  
        Examples include:

        \begin{itemize}
            \item \textbf{Typhoon Noru (2022):} Recorded wind gusts of level 14, causing severe damage in Central Vietnam.
            \item \textbf{Typhoon Yagi (2024):} Reached level 13 winds when approaching the Thua Thien–Hue coastline.
        \end{itemize}

        \begin{figure}[H]
            \centering
            \includegraphics[width=0.75\textwidth]{pictures/chapter4/c4_p2_DistributionStorm.png}
            \caption{Distribution of maximum wind speed of Typhoon Linda (1997)\protect\footnotemark}
            \label{fig:typhoon_linda}
        \end{figure}
        \footnotetext{Hiroshi Takagi (2019). ``Distribution of the storm surge induced by Typhoon Linda''. \textit{ResearchGate}.}

        \subsection{Recent Trends}

        \hspace*{0.6cm}Under the influence of climate change, typhoons have become stronger, more unpredictable, and tend to occur later in the year.  
        According to the Ministry of Natural Resources and Environment (MONRE, 2024), the number of strong typhoons (category $\geq$ 12) has slightly increased over the past two decades. Between 2020--2022, Central Vietnam experienced 4--5 strong typhoons per year.

        \begin{figure}[H]
            \centering
            \includegraphics[width=0.8\textwidth]{pictures/chapter4/c4_p3_StormTrack.png}
            \caption{Tracks of several tropical storms in the East Sea (example: Typhoon Kalmaegi)\protect\footnotemark}
        \end{figure}
        \footnotetext{Source: The Independent (2025). ``Tropical Storm Paths over the South China Sea''.}

        \subsection{Summary}

        \hspace*{0.6cm}In summary, the key characteristics of storms in Vietnam are as follows:

        \begin{itemize}
            \item On average, 10–12 tropical cyclones form annually in the East Sea, with 5–7 directly affecting Vietnam’s mainland.
            \item The main typhoon season runs from June to November, peaking in September–October.
            \item Common wind intensity ranges from level 9–12 (25–40 m/s).
            \item In recent years, storms have shown increasing intensity, heavier rainfall, and more complex trajectories due to climate change.
        \end{itemize}
    \section{Describe the rainfall characteristics of Vietnam by region}

        \hspace*{0.6cm}Vietnam has diverse topography, including plains, coastal areas, highlands, and mountains, which leads to significant regional variations in rainfall. The country can be divided into several rainfall zones with distinct seasonal and spatial characteristics.

        \subsection{Annual Rainfall Distribution}

            \hspace*{0.6cm}Based on data from the Ministry of Natural Resources and Environment (MONRE, 2024) and research by JICA (2020), Vietnam is commonly divided into seven rainfall regions:

            \begin{itemize}
                \item R1: Northwest
                \item R2: Northeast
                \item R3: Red River Delta
                \item R4: North Central Highlands
                \item R5: South Central Coast
                \item R6: Central Highlands
                \item R7: Mekong Delta
            \end{itemize}

            \hspace*{0.6cm}Annual rainfall varies significantly between regions:

            \begin{itemize}
                \item Northern mountainous areas: up to 4,000 mm/year
                \item Central coastal areas: 1,800--3,000 mm/year
                \item Southern regions: 1,900--2,600 mm/year
            \end{itemize}

            \begin{figure}[H]
                \centering
                \includegraphics[width=0.7\textwidth]{pictures/chapter4/c4_p4_vn_rainfall_map.png}
                \caption{Annual rainfall distribution in Vietnam and the seven rainfall regions\protect\footnotemark}
            \end{figure}
            \footnotetext{Hoa Thi Tran, James B. Campbell \& Randolph Wynne. (2019). ``Drought and Human Impacts on Land Use and Land Cover Change in a Vietnamese Coastal Area''.}
        \subsection{Rainy Season by Region}

            \hspace*{0.6cm}Rainfall seasonality differs across regions:

            \begin{itemize}
                \item \textbf{Northern Vietnam (R1--R3):} Rainy season from April to October, peak in summer months.  
                \item \textbf{Central Vietnam (R4--R5):} Rainfall occurs mainly from September to December, influenced by tropical cyclones and northeast monsoon.  
                \item \textbf{Southern Vietnam (R6--R7):} Rainy season from May to October/November, with the rest of the year being dry.
            \end{itemize}

            \begin{figure}[H]
                \centering
                \includegraphics[width=0.7\textwidth]{pictures/chapter4/c4_p5_vn_rainy_season_map.png}
                \caption{Rainy season distribution across Vietnam\protect\footnotemark}
            \end{figure}
            \footnotetext{Gummadi, S., Dinku, T., Shirsath, P., \& Kadiyala, D. M. (2022). ``Evaluation of multiple satellite precipitation products for rain-fed maize production systems over Vietnam''.}

            \subsection{Rainfall Intensity and Variability}

            \hspace*{0.6cm}Rainfall in Vietnam is often highly concentrated:

            \begin{itemize}
                \item In many regions, 70--90\% of the annual rainfall occurs during the rainy season or associated with tropical depressions and typhoons.
                \item Coastal central areas are prone to heavy rainfall and flooding due to typhoons.
                \item Northern mountainous regions experience highly variable rainfall, including both continuous light rain and short intense showers, which can trigger landslides.
            \end{itemize}

            \begin{figure}[H]
                \centering
                \includegraphics[width=0.7\textwidth]{pictures/chapter4/c4_p6_vn_rainfall_chart.png}
                \caption{Monthly average rainfall of representative regions in Vietnam\protect\footnotemark}
            \end{figure}
            \footnotetext{Tung Tran Thanh. (2011). ``Morphodynamics of seasonally closed coastal inlets at the central coast of Vietnam''. \textit{Thesis}.}

        \subsection{Summary}

        \hspace*{0.6cm}In summary, the rainfall characteristics of Vietnam are as follows:

        \begin{itemize}
            \item Vietnam exhibits strong regional differentiation in rainfall. Northern regions have early and summer-concentrated rains; southern regions have a mid-year rainy season with a clear dry period; central regions experience late rains associated with tropical depressions.
            \item Annual rainfall is high, especially in mountainous and coastal areas, ranging from 1,400 mm/year in some plains to over 4,000 mm/year in highlands.
            \item Rainfall variability and concentration have significant implications for water resources planning, flood control, and climate adaptation strategies.
        \end{itemize}
    \section{Describe the impact of storms on landslide and flood problems}

        \hspace*{0.6cm}When tropical cyclones make landfall in Vietnam, their impacts extend beyond strong winds and storm surges to include **heavy rainfall**, which often causes two major hazards: **flooding** and **landslides**. Due to Vietnam's diverse topography—including plains, coastal regions, and mountainous areas—the country is highly vulnerable to these hazards during storms.

        \subsection{Impact on Flooding}

        \hspace*{0.6cm}Recent typhoons, such as Typhoon Yagi (2024), brought severe flooding in northern Vietnam, submerging streets and isolating communities \cite{guardian2024}.  
        Flooding occurs when intense rainfall overwhelms urban drainage systems or river capacities, affecting low-lying and delta regions like the Red River Delta and Mekong Delta. Consequences include inundated homes, disrupted transportation, crop damage, power outages, and significant economic losses.

        \begin{figure}[H]
            \centering
            \includegraphics[width=0.75\textwidth]{pictures/chapter4/c4_p7_flood_yagi.png}
            \caption{Severe flooding in Northern Vietnam during Typhoon Yagi (2024)\protect\footnotemark}
        \end{figure}
        \footnotetext{The Guardian, 2024. \url{https://www.theguardian.com/world/article/2024/sep/09/typhoon-yagi-vietnam-weather-warnings-death-toll-floods-landslides}}

        \subsection{Impact on Landslides}

        \hspace*{0.6cm}In mountainous and highland regions, heavy and prolonged rainfall saturates soil, reducing cohesion and triggering **landslides and rockfalls**. For example, Typhoon Prapiroon (2024) caused major landslides along National Route 6, Hoa Binh Province, blocking traffic \cite{sggp2024}.  
        Rainfall combined with steep slopes and narrow valleys often generates flash floods, exacerbating landslide risks. Consequences include destroyed homes, loss of life, and blocked transportation routes, making rescue operations difficult.

        \begin{figure}[H]
            \centering
            \includegraphics[width=0.75\textwidth]{pictures/chapter4/c4_p8_landslide_phap.png}
            \caption{Landslide caused by heavy rainfall in northern Vietnam\protect\footnotemark}
        \end{figure}
        \footnotetext{SGGP, 2024. \url{https://en.sggp.org.vn/typhoon-prapiroon-triggers-landslides-cuts-off-traffic-in-northern-vietnam-post111370.html}}

        \subsection{Link Between Storms, Floods, and Landslides}

        \hspace*{0.6cm}The relationship between storms, floods, and landslides is direct:

        \begin{itemize}
            \item Storms bring both **strong winds** and **intense rainfall**. Short-duration heavy rainfall leads to rapid river rises and flooding.
            \item Simultaneously, water infiltration on slopes reduces soil stability, causing landslides.
            \item Coastal and low-lying areas are primarily affected by flooding, whereas mountainous regions are prone to landslides and flash floods.
            \item Climate change increases the intensity and frequency of extreme rainfall events and strong storms, exacerbating these hazards.
        \end{itemize}

        \subsection{Summary}

        \hspace*{0.6cm}In summary, storms in Vietnam not only bring wind and storm surges but also **flooding and landslides**, which are major hazards. Lowland and urban areas experience severe flooding, while mountainous regions face landslides and flash floods. Early warning systems, risk zoning, and resilient infrastructure are essential to mitigate the impacts of these natural hazards.


