\chapter{BIOSPHERE}

\section{Biological Weathering by Plant Roots in Rock Fractures Over Time}

\subsection{Overview and Coupled Mechanical-Chemical Mechanisms}

Biological weathering refers to the actions of living organisms---here the roots---that weaken, fracture and alter minerals and rock near the surface. On bare rock, roots act persistently and effects accumulate over time.

\textit{Mechanically}, young roots enter moist fissures, then expand in diameter and wedge the fracture walls. Wet-dry cycles cause swelling-shrinkage, inducing mechanical fatigue and widening micro-cracks.

\textit{Chemically}, the rhizosphere releases organic acids and chelators that promote ion exchange and dissolution: clay-coated minerals in the silicates and other minerals. The substrate is softened and becomes easier to pry apart.

\textit{Mutual amplification:} mechanics opens fresh mineral surfaces for chemistry to attack; chemistry weakens minerals so mechanics becomes more effective---forming a seasonal, long-term feedback loop.

\subsection{Temporal Progression}

A seed that falls into a moist crack germinates; the taproot grows downward seeking water while lateral roots explore horizontally. Secondary root branching and thickening increases root diameter and wedging pressure perpendicular to the fracture. Wet-dry cycles further fatigue the rock and widen micro-cracks.

At the same time, root exudates dissolve and alter minerals right at the crack margins. Because the rock is ``softened'', each subsequent wedging is more efficient. After several rainy-dry seasons the aperture enlarges; fine mineral debris accumulates and mixes with humus to form young soil within the crack.

Over longer periods, formerly isolated cracks connect into a network and a weathering rind develops. In carbonates, margins tend to be rounded by dissolution; in granite, feldspars alter to clay leaving veins that infill the fracture.

\subsection{Factors Controlling Process Intensity}

\textbf{Lithology/minerals:} carbonates dissolve more readily (chemistry-dominated); granite rich in feldspar alters to clay; dense metamorphic rocks progress slowly.

\textbf{Fracture geometry:} narrow--long cracks concentrate wedging stress; open--ventilated cracks retain moisture and aid germination.

\textbf{Climate--hydrology:} humid climates with a large wet--dry amplitude accelerate both mechanics and chemistry; frost further enhances cracking where present.

\textbf{Biology:} vigorous rooting and mycorrhizae speed chemical alteration; lichens often initiate weathering on bare rock.

\section{Estuarine Mangrove Characteristics and Their Role in Shoreline Protection}

\subsection{Characteristics of Estuarine Mangroves}

Estuarine mangroves occupy the river-sea transition, where saline water mixes with freshwater and tidal forcing dominates. Beyond high biodiversity, mangroves serve as a natural "green shield" against climate change and coastal dynamics.

\textit{Structure \& biology:} mangrove trees possess specialized adaptations-pneumatophores for gas exchange, stilt-prop roots that stabilize stems and trap sediment, and salt - excluding/excreting leaves. Zonation follows elevation and salinity: wave-exposed fringes are typically Rhizophora/Avicennia; more sheltered interiors host Bruguiera, Sonneratia, etc.; beneath the canopy, halophytic herbs and biofilms bind fine particles.

\textit{Functions:} mangroves dissipate wave energy, trap sediment to raise tidal flats, limit erosion and reduce saline intrusion inland. Root surfaces and organic mud also provide natural filtration, retaining suspended solids and hosting microbes that degrade organics.

\subsection{Mangroves at Cua Lap}

At Cua Lap (Ba Ria--Vung Tau), mangroves spread on muddy--sandy tidal flats under a semi-diurnal tide and monsoonal waves. The seaward fringe is dominated by Rhizophora and Avicennia with robust stilt and aerial roots; inland zones include Bruguiera and Sonneratia alongside halophytes and biofilms bind fines.

A dense creek network enhances water exchange; headland shoals where roots trap sediment form "root platforms" that gradually advance seaward. During windy seasons the Rhizophora/Avicennia fringe is critical for dissipating waves along the river-mouth--sandy-shore corridor.

\subsection{Role of Mangroves in Shoreline Protection}

\textbf{Wave damping -- hazard reduction:} canopies, trunks and roots increase hydraulic roughness, lower flow velocity and dissipate wave energy before it reaches hard structures---thus reducing erosion, bank failure and damage.

\textbf{Sediment retention -- elevation gain:} reduced velocities promote settling of suspended mud and sand around roots, building platforms and raising tidal-flat elevation; shorelines stabilize or even advance where sediment supply is adequate.

\textbf{Salinity control \& water quality:} mangrove belts attenuate tidal oscillations and saline intrusion; root--mud surfaces adsorb particulates and host microbes that degrade organics---water becomes clearer.

\textbf{Infrastructure \& livelihoods:} by cutting waves and fostering accretion, mangroves protect dikes, roads and settlements, while sustaining fisheries, eco-tourism and blue-carbon storage.

\subsection{Conditions for Sustained Effectiveness and Challenges}

\textbf{Prerequisites:} a belt that is wide and continuous, species zonation matching tidal elevation, steady sediment supply and acceptable water quality.

\textbf{Challenges:} land conversion, pollution, over-exploitation, sea-level rise and stronger storms; mis-zoned planting leads to poor survival.

\textbf{Restoration tips:} apply nature-based solutions (soft sediment traps, bamboo wave-breaks) to help the forest capture sediment during early recovery.
