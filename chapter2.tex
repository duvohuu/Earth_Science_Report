\chapter{HYDROSPHERE}

    \section{Groundwater in Coastal Sand Dunes: Recharge Sources, Physicochemical Characteristics, and Saltwater Intrusion in Wells. Illustration at Cua Lap}
        \subsection{Groundwater in Coastal Sand Dunes}
        \hspace*{0.6cm}The coastal sand dune/ridge area has distinct geological characteristics that affect the groundwater in these regions.
        \begin{itemize}
            \item \textbf{Geological Characteristics:}
                \begin{itemize}
                    \item \textbf{Topography:} The coastal sand dunes and ridges at Cua Lap are deposited by wind and sea currents.
                    \item \textbf{Composition:} The main component is fine to medium sand, interspersed with small gravel and marine organism shells. There are also patches of black sand containing Ilmenite.
                    \item \textbf{Aquifer:} Unconsolidated sand, possessing good permeability and significant thickness.
                    \item \textbf{Water Table Depth:} The groundwater level is not deep, depending on the location and height of the sand dune. It is often lower than sea level, leading to a high risk of saltwater intrusion.
                \end{itemize}

            \item \textbf{Recharge Source:}

                \begin{itemize}
                    \item The groundwater in this area is primarily recharged by rainwater.
                    \item In the coastal dune area, the sand layer has high porosity and permeability. When it rains, most of the water percolates into the ground, accumulating in sand layers at various depths, forming an aquifer.
                    \item Surface wells (dug wells) are formed by rainwater that accumulates on the dunes and is held by an impermeable layer.
                    \item Below and surrounding the sand dune is saltwater that has infiltrated from the sea. Therefore, as rainwater falls and seeps into the ground, being less dense (lighter) than saltwater, it forms a freshwater lens ``suspended'' above the saltwater body.
                \end{itemize}

            \item \textbf{Physicochemical Characteristics:}

                \begin{itemize}
                    \item The surface well has a mouth diameter of approximately 1.3 m, the distance from the well mouth down to the water surface is approximately 1.9 m, and the water depth in the well is about 1 m.
                    \item \textbf{Water properties:} Colorless, odorless, tasteless.
                    \item \textbf{Physicochemical characteristics:}
                    \begin{itemize}
                        \item \textbf{Salinity:} The well water is fresh, originating mainly from rainwater, thus having low salinity.
                        \item \textbf{pH:} The well water originates from rainwater, which has dissolved CO$_2$ from the atmosphere, and it also percolates through a mineral-poor sand layer. The pH tends to be neutral.
                    \end{itemize}
                    \item \textbf{Saltwater Intrusion in Wells:} Influenced by proximity to the sea, reduced rainfall during the dry season, and tides, saltwater from the sea can intrude into the freshwater layer. This alters the water's physicochemical characteristics, thereby affecting the daily life and activities of the residents.
                \end{itemize}
        \end{itemize}

        \subsection{Illustration at Cua Lap}
            \begin{itemize}
                \item \textbf{Coordinates:} 10°40.8'N, 107°16.9'E
                \item \textbf{Date:} 01/11/2025
                \item \textbf{Weather:} Light sun, foggy.
                \item \textbf{Characteristics:}
                    \begin{itemize}
                        \item Sand dune deposited by wind and sea currents.
                        \item Fine sand, containing the black mineral ilmenite.
                        \item Phenomena of erosion and accretion (deposition) due to wind and wave action.
                        \item Dug well parameters: 1.3 m mouth diameter, 1.9 m height from mouth to water surface, 1 m water level (depth).
                        \item Water is fresh (tasteless), colorless, and odorless.
                    \end{itemize}

                    \begin{figure}[H]
                        \centering
                        \includegraphics[width=0.65\textwidth]{pictures/chapter2/c2_p1_CuaLap_well.png}
                        \caption{Photograph of a perched groundwater well at Cua Lap}
                        \label{fig:cua_lap_well}
                    \end{figure}

                    \begin{figure}[H]
                        \centering
                        \includegraphics[width=0.65\textwidth]{pictures/chapter2/c2_p2_CuaLap_Salt.png}
                        \caption{Coastal sand dunes at Cua Lap}
                        \label{fig:cua_lap_dunes}
                    \end{figure}
        \end{itemize}

    \section{Description of Riverbank and Coastal Erosion Processes}

    \hspace*{0.6cm}Erosion is the process by which soil and rock are worn away and transported from surfaces by natural factors such as water and wind. For riverbank and coastal erosion, water is the primary agent.

    \subsection{Riverbank Erosion}

    \begin{itemize}
        \item \textbf{Agent:} Primarily river flow (current) and water level.
        \item \textbf{Process and Impact:} Hydraulic action, where the force of the flowing water directly impacts the riverbank, dislodging and washing away soil and rock particles.
        \item Additionally, abrasion occurs as materials (such as soil and rocks) carried by the water rub against the banks and bed, wearing them down.
    \end{itemize}

    \begin{figure}[H]
        \centering
        \includegraphics[width=0.7\textwidth]{pictures/chapter2/c2_p3_KonRiver.png}
        \caption{Bank erosion along the Kon River}
        \label{fig:river_erosion}
    \end{figure}

    \subsection{Coastal Erosion}

    \begin{itemize}
        \item \textbf{Agent:} Sea waves.
        \item \textbf{Process:} Waves crash against the shore (sand or cliffs). Waves can compress air within cracks and fissures; as the wave recedes, the pressure release creates small ``explosions,'' shattering the rock.
        \item Simultaneously, as waves strike, they carry sand and gravel, which abrade (scour) the surface.
        \item The result of erosion is the gradual retreat of the coastline inland, creating landforms such as sea cliffs, wave-cut notches, and sea caves.
    \end{itemize}

    \begin{figure}[H]
        \centering
        \includegraphics[width=0.7\textwidth]{pictures/chapter2/c2_p4_CoastalErosion.png}
        \caption{Partial coastal erosion process}
        \label{fig:coastal_erosion}
    \end{figure}