\chapter*{INTRODUCTION}
    \addcontentsline{toc}{chapter}{INTRODUCTION}

    \section*{Purpose of Conducting a Field Report}

    \hspace*{0.6cm}The main purpose of preparing a field report is to systematically document observations, measurements, and analyses made directly at a study site. A well-prepared field report serves as a comprehensive record of empirical data, collecting accurate and detailed information about geological features, rock types, structures, and environmental conditions observed in the field. This documentation provides essential evidence for interpreting geological processes, understanding the geological history of the area, and establishing relationships between different rock formations and structural features.

    Beyond data collection, field reports play a crucial role in developing practical skills in field observation, geological mapping, and scientific data interpretation. They facilitate clear communication of findings with peers, instructors, and researchers, allowing others to verify results, draw upon the documented observations, and build upon existing knowledge. Furthermore, field reports serve as valuable references for identifying patterns, anomalies, or areas of particular interest that warrant further investigation and detailed research.

    In summary, a field report ensures that observations made in situ are systematically recorded, analyzed, and communicated, forming a foundation for scientific study and decision-making in earth sciences.

    \section*{Task Assignment Table for Group Members}
    
    \begin{table}[h]
        \centering
        \begin{tabular}{|c|l|c|}
            \hline
            \textbf{Student's ID} & \textbf{Student's name} & \textbf{Task} \\
            \hline
            2210497 & Duong Quang Duy & Chapter I\\
            \hline
            2210350 & Dao Trong Chan & Chapter II\\
            \hline
            2210647 & Tran Quang Dao & Chapter III\\
            \hline
            2210604 & Vo Huu Du & Chapter VI\\
            \hline
            2313492 & Nguyen Thanh Toan & Chapter V\\
            \hline
        \end{tabular}
        \caption{Task assignment for group members}
        \label{tab:task_assignment}
    \end{table}
    
    \section*{Natural, Economic, and Social Characteristics of the Study Area}

    \subsection*{Natural Characteristics}

    \hspace*{0.6cm}Ba Ria – Vung Tau is located in the Southeast region of Vietnam, bordering Dong Nai and Binh Thuan provinces, as well as Ho Chi Minh City, while facing the East Sea to the southeast. The area exhibits diverse terrain characterized by a long coastline, hills, mountains, and plains, with the highest peak being Nui Chua at 503 meters above sea level. Vung Tau is particularly renowned for its scenic beaches, including Bai Truoc and Bai Sau.
    
    The region experiences a tropical monsoon climate with two distinct seasons: a rainy season extending from May to October and a dry season from November to April. The average annual temperature is approximately 27°C, creating favorable conditions for both tourism and agriculture. The hydrological system is relatively abundant, featuring notable waterways such as the Dinh River and Ray River, along with an extensive network of canals connected to the sea, which play crucial roles in irrigation, transportation, and aquaculture activities.

    \subsection*{Economic Characteristics}

    \hspace*{0.6cm}Ba Ria – Vung Tau serves as an important industrial center in southern Vietnam, with particular emphasis on the oil and gas sector, chemical production, and seafood processing industries. Major industrial zones, including Phu My and Cai Mep – Thi Vai, contribute significantly to the regional and national economy. The tourism sector represents another key economic pillar, with numerous attractive destinations such as Bai Sau, Ho Coc, Long Hai, and Con Dao drawing millions of domestic and international visitors annually.
    
    Agriculture remains an important economic component, with the cultivation of rice, rubber, pepper, and various fruit trees. Aquaculture activities, particularly cage fish and shrimp farming, make substantial contributions to the local economy. The province benefits from a well-developed transportation infrastructure, featuring strategically important seaports such as Cai Mep – Thi Vai Port and Vung Tau Port, complemented by an extensive road network that connects neighboring provinces and Ho Chi Minh City, facilitating trade and economic development.

    \subsection*{Social Characteristics}

    \hspace*{0.6cm}The population of Ba Ria - Vung Tau is approximately 1.1 million people, representing diverse ethnicities and cultures, with the main ethnic groups including Kinh, Hoa, Khmer, and Chăm. The provincial education system receives substantial investment, encompassing numerous schools, training centers, and institutions of higher learning such as Ba Ria – Vung Tau University, with educational quality showing steady improvement over recent years.
    
    The healthcare infrastructure is well-developed, comprising numerous hospitals, medical centers, and clinics, with Ba Ria Hospital and Le Loi Hospital serving as major healthcare facilities. The region possesses rich cultural heritage, manifested through traditional festivals and historical sites including the Temple of Saint Tran Hung Dao, Long Son Great House, and various folk festivals, while maritime culture forms an integral part of community life.
    
    The quality of life for residents has demonstrated continuous improvement, supported by increasingly adequate and modern public services and social amenities. Economic development has contributed to a significant reduction in the poverty rate, and living standards continue to rise, reflecting the overall socioeconomic progress of the province.